\documentclass{article}               % use any documentclass you like
\usepackage{graphicx,amssymb}         % use any packages you like
\usepackage{questiontex}              % include the questiontex package

                                      % you may include macros of your own
                                      % (but limits apply for Moodle import)
\begin{document}

%\hidesolution                        % uncomment this line to hide solution
                                      % and produce printer's copy of the test

\intro{%
    \noindent 
    An intro for the multiple-choice test. Write anything you want here.
    \bigskip\hrule\bigskip
}

% 1. Hello, world!

\question{Our first question.}
\true{A true answer.}
\feedback{Feedback for those who selected this correct answer (optional).}
\false{A false answer.}
\feedback{Feedback for those who selected this incorrect answer (optional).}
\true{Feedbacks are optional.}
\explanation{Explanation of how to arrive at the correct solution.
  Explanations are optional.}

% 2. LaTeX support

\question{%
    What \LaTeX\ is allowed?
}
\false{%
Labels and references.%
}
\true{%
    Besides labels and references you can use any \LaTeX. Anywhere.
    $$\prod_{p\in{\bf \mathbb{P}}}\frac{1}{1-\frac{1}{p^2}}=\frac{\pi^2}{6}$$
}
\true{%
    If you plan on importing to Moodle, write plain text and use LaTeX only in
    math mode.  We are sorry, but Moodle's filters only handle math mode.
}


% 3. Encouraging the use of feedbacks and explanations

\question{%
    Why bother with feedbacks and explanations, when they're optional?
}
\false{Feedbacks are just unnecessary effort.}
\feedback{%
    Immediate feedback is one of the great strengths of interactive
    multiple-choice tests.
}
\true{%
    Your students will appreciate your efforts.
}

%% 4. How to use images
%
%\question{%
%    Images must be present in {\tt eps} as well as {\tt pdf} format.
%    \begin{center}
%        \includegraphics[width=5cm]{myImages/cube}
%    \end{center}
%}
%\true{%
%    You can use images anywhere.
%    \begin{center}
%    \includegraphics[width=5cm]{myImages/grafSinPlusCos}
%    \end{center}
%}
%\feedback{%
%    This is a feedback with an image.
%    \begin{center}
%    \includegraphics[width=5cm]{myImages/pote5}
%    \end{center}
%}

\end{document}
