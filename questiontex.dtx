% \iffalse meta-comment
% !TEX program = pdfLaTeX
%<*internal>
\iffalse
%</internal>
%<*readme>
========================================================================
questiontex --- write multiple-choice tests in LaTeX
E-mail: echo@math.ethz.ch
Released under the LaTeX Project Public License v1.3c or later
See http://www.latex-project.org/lppl.txt
========================================================================

I. OVERVIEW

The questiontex package contains a collection of LaTeX macros for writing
multiple choice tests. The LaTeX sources can be processed by 'latex' in 
order to produce a printer's copy and standard solution. 
Interactive versions are obtained through import into the Moodle Learning 
Management System via a dedicated plugin or through in-house software 
for classroom assessment tests developed at ETH Zurich.

========================================================================

II. INSTALLATION REQUIREMENTS

A LaTeX distribution supporting the 'pdftex' and 'pdflatex' commands.

========================================================================

III. INSTALLATION AND GETTING STARTED

 A) Produce questiontex.sty
      pdftex questiontex.dtx
      
 B) Typeset documentation      
      pdflatex questiontex.dtx

 C) Typeset example test
      pdflatex example.tex
      
 D) Add questiontex package to your TeX distribution (optional)
    by copying the questiontex folder into
      <tex-root-directory>/texmf/tex/latex

========================================================================

IV. CONTACT

This software was written in the projects LEMUREN and nemesis at the 
department of mathematics of ETH Zurich. For further information, please
contact echo@ethz.ch.

%</readme>
%<*internal>
\fi
\def\nameofplainTeX{plain}
\ifx\fmtname\nameofplainTeX\else
  \expandafter\begingroup
\fi
%</internal>
%<*install>
\input docstrip.tex
\keepsilent
\askforoverwritefalse
\preamble
========================================================================
questiontex --- write multiple-choice tests in LaTeX
E-mail: nemesis@math.ethz.ch
Released under the LaTeX Project Public License v1.3c or later
See http://www.latex-project.org/lppl.txt
========================================================================

\endpreamble
\postamble

Copyright (C) 2014 by Project LEMUREN, ETH Zurich

This work may be distributed and/or modified under the
conditions of the LaTeX Project Public License (LPPL), either
version 1.3c of this license or (at your option) any later
version.  The latest version of this license is in the file:

http://www.latex-project.org/lppl.txt

This work is "maintained" (as per LPPL maintenance status) by
project nemesis, ETH Zurich <echo@math.ethz.ch>

\endpostamble
\usedir{tex/latex/questiontex}
\generate{
  \file{\jobname.sty}{\from{\jobname.dtx}{package}}
}
%</install>
%<install>\endbatchfile
%<*internal>
\usedir{source/latex/questiontex}
\generate{
  \file{\jobname.ins}{\from{\jobname.dtx}{install}}
}
\nopreamble\nopostamble
\usedir{doc/latex/questiontex}
\generate{
  \file{README.txt}{\from{\jobname.dtx}{readme}}
}
\ifx\fmtname\nameofplainTeX
  \expandafter\endbatchfile
\else
  \expandafter\endgroup
\fi
%</internal>
%<*package>
\NeedsTeXFormat{LaTeX2e}
\ProvidesPackage{questiontex}[2017/08/16 v0.1.1 Package for multiple-choice tests]
%</package>
%<*driver>
\documentclass{ltxdoc}
\usepackage[T1]{fontenc}
\usepackage{lmodern}
\usepackage{\jobname}
\usepackage[numbered]{hypdoc}
\EnableCrossrefs
\CodelineIndex
\RecordChanges
\begin{document}
  \DocInput{\jobname.dtx}
\end{document}
%</driver>
% \fi
%
% \CharacterTable
%  {Upper-case    \A\B\C\D\E\F\G\H\I\J\K\L\M\N\O\P\Q\R\S\T\U\V\W\X\Y\Z
%   Lower-case    \a\b\c\d\e\f\g\h\i\j\k\l\m\n\o\p\q\r\s\t\u\v\w\x\y\z
%   Digits        \0\1\2\3\4\5\6\7\8\9
%   Exclamation   \!     Double quote  \"     Hash (number) \#
%   Dollar        \$     Percent       \%     Ampersand     \&
%   Acute accent  \'     Left paren    \(     Right paren   \)
%   Asterisk      \*     Plus          \+     Comma         \,
%   Minus         \-     Point         \.     Solidus       \/
%   Colon         \:     Semicolon     \;     Less than     \<
%   Equals        \=     Greater than  \>     Question mark \?
%   Commercial at \@     Left bracket  \[     Backslash     \\
%   Right bracket \]     Circumflex    \^     Underscore    \_
%   Grave accent  \`     Left brace    \{     Vertical bar  \|
%   Right brace   \}     Tilde         \~}
%
% \DoNotIndex{\#,\$,\%,\&,\@,\\,\{,\},\^,\_,\~,\ }
% \DoNotIndex{\advance,\begingroup,\catcode,\closein}
% \DoNotIndex{\closeout,\day,\def,\edef,\else,\empty,\endgroup}
%
% \GetFileInfo{\jobname.sty}
%
% \title{The \textsf{questiontex} package\thanks{This document
% corresponds to \textsf{questiontex}~\fileversion,
% dated \filedate.}}
% \author{Project nemesis, ETH Zurich \\ \texttt{echo@math.ethz.ch}}
% \date{Released \filedate}
%
% \maketitle
%
% \changes{v0.1}{2014/03/17}{Initial packaged version}
% \changes{v0.1.1}{2017/08/16}{Standalone github repo, update contact details}
%
% \begin{abstract}
%  The \textsf{questiontex} package contains a collection of \LaTeX\ macros
%  for writing multiple choice tests.
%  The \LaTeX\ sources can be processed by \verb|latex| in order to produce a 
%  printer's copy and standard solution. Interactive versions are 
%  obtained through import into the 
%  \href{https://moodle.org/}{Moodle Learning Management System} via 
%  a \href{https://moodle.org/plugins/qformat_qtex}{dedicated plugin}
%  or through in-house software for classroom assessment tests
%  developed at ETH Zurich.
% \end{abstract}
%
% \tableofcontents
% \pagebreak
% \PrintChanges
% 
% \section{Introduction}
% Question\TeX\ is a collection of \LaTeX\ macros that enables authors to
% write multiple-choice tests.
% The \LaTeX\  sources can then be processed in order to
% \begin{itemize}
%   \item create a high quality printer's copy with standardized
%   layout
%   \item create a standard solution, including additional feedback
%   \item create an interactive classroom assessment test (CAT)
%   \item import the questions into the Moodle Learning Management System 
%   \item \ldots your idea here!
% \end{itemize}
% 
% \noindent A basic multiple choice question may be written as
% 
% \begin{center}
%   \begin{minipage}[b]{0.75\textwidth}
%     \begin{verbatim}
% \question{The square root of two is \ldots}
% \false{a rational number.}
%   \feedback{Try to represent it as a quotient of integers!}
% \true{a real number.}
% \false{an imaginary number.}
%      \end{verbatim}
%   \end{minipage}
% \end{center}
% \hfill
% \vspace{-2em}
%
% \noindent and the corresponding standard solution is typeset by \LaTeX\ as
%
% \begin{center}
%   \begin{minipage}[b]{0.75\textwidth}
%     \question{The square root of two is \ldots}
%         \false{a rational number.}
%             \feedback{Try to represent it as a quotient of integers!}
%         \true{a real number.}
%       \false{an imaginary number.}
%   \end{minipage}
% \end{center}
% \hfill
%
% For an overview of the macros available   
% check out the \verb|example.tex| that comes with this package. 
% A more detailed command specification is found in section
% \ref{seccommand} of this document.
% 
% \section{Deployment mechanisms}
% 
% \subsection{Creation of a solution}
% Use \verb|latex| or \verb|pdflatex| to create a standard solution of your
% test in \verb|.dvi| or \verb|.pdf| format. 
% The solution indicates, which answers are true.
% If provided, it also includes feedback on individual answers and/or a
% general explanation of the question.
% 
% \subsection{Creation of a printer's copy}
% Use the |\hidesolution| macro to hide all solution meta data
% and produce a printer's copy of your test,
% ready to be handed out to students.
% 
% \subsection{Creation of an interactive online-test}
% This feature is not fully automated yet, but a user friendly upload mechanism
% with graphical user interface is coming up soon (2010).
% At the moment, you can send us an email with your source 
% and preferred grading rules and we will set up the
% test on our system.
% 
% \subsection{Import of questions into Moodle}
% A Question\TeX\ plugin for the Moodle Learning Management System
% is available from \verb|moodle.org/plugins|.
% It provides a Question\TeX\ import/export format for  multiple
% choice questions. 
%
% After the installation you will be able to upload your
% Question\TeX\ sources directly to Moodle. In case your sources include
% image files, you may create a zip archive with all the relevant
% files and upload the zip instead.
%
% \emph{Note:} Since Moodle does not come with a built-in \LaTeX\ distribution, 
% certain restrictions must be obeyed in order to ensure proper display of your questions
% in Moodle (detailed information in section \ref{secrestr}).
% 
% \section{Command reference}\label{seccommand}
% All commands are sorted alphabetically.
% 
% \subsection{Writing questions}
% This is a selection of the available commands for writing questions. In
% order to see some examples for questions, have a look at the 
% \verb|example.tex| that comes with this package.
% 
% \DescribeMacro{\explanation} May be used to outline an approach to the 
% solution. If present, this command will normally be placed at the very end of
% a question. When the questions are deployed in a static context, the
% visibility of the explanation may be controlled by |\hidesolution|.
% 
% \DescribeMacro{\false} Contains a wrong answer.\\
% |\false{Some wrong answer.}|
% 
% \DescribeMacro{\feedback} If you want to give a feedback to a specific answer,
% you may do so by using the feedback-command \emph{after the
% answer}.\\ This is especially useful, when the questions are deployed
% in an interactive context, since the feedback to a student will then
% depend on his or her answers. In a static context the display of feedbacks can
% be controlled by |\hidesolution|.
% 
% \DescribeMacro{\includegraphics} Should be used to include images. \LaTeX\
% compilation supports the |eps| format, while Pdf\LaTeX\ supports |png|, |jpg|,
% |tiff| and  |pdf|.\\
% Images may be included everywhere, where ordinary text can appear, i.e. they
% may be part an answer, a feedback or the like:\\
% |\feedback{Correct! \includegraphics[height=1cm]{smiley.png}}|
% 
% \DescribeMacro{\intro} Inserts arbitrary text that is not an argument to 
% another command into the quiz.
% 
% \DescribeMacro{\keepme} Same as |\intro|.
% 
% \DescribeMacro{\question} 
% Defines the question text.
% An identifier of the
% question may be supplied via the optional parameter. The identifier must
% consist only of letters of the English alphabet and the underscore \_ :\\
% |\question[Identifier_1]{Now, ain't that easy?}|
% 
% \DescribeMacro{\questionSc}
% Like \verb|\question|, but specifies that this question has exactly
% one correct answer.
% This is relevant in interactive contexts, where students may be restricted
% to selecting a single answer only.
% 
% \DescribeMacro{\true} Contains a true answer. A question may have multiple
% true answers.\\
% |\true{My true answer}|
% 
% %%%%%%%%%%%%%%%%%%%%%%%%%%%%%%%%%%%%%%%%%%
% \subsection{Setting global properties}
% The following commands define global properties of the quiz. They should
% appear \emph{before} the first question.
% 
% \DescribeMacro{\hidesolution} 
% Hides all solution-related meta data, 
% i.\,e., only questions and answers are shown.
% 
% %%%%%%%%%%%%%%%%%%%%%%%%%%%%%%%%%%%%%%%%%%%%%%%%%%%%%%%%%%%%%%%%%%%%%%%%%%%%%%
% \section{Appendix}
% 
% %%%%%%%%%%%%%%%%%%%%%%%%%%%
% \subsection{Restrictions for Moodle import}\label{secrestr}
% As of today (2014), there are two main display mechanisms in Moodle
% \begin{enumerate}
%   \item The browser processes HTML. It does not understand \LaTeX\ at all.
%   \item Plug-ins like mimeTeX, MathJax or JsMath process
%     formulae that are enclosed by
%   certain delimiters and convert them into graphics. They do understand
%   appreciable parts of \LaTeX\, but not everything.
% \end{enumerate}
% 
% Since plain text and images are processed directly by the browser, while
% formulae are processed by the plug-in, different rules apply.
% 
% \subsubsection{Plain text}
% Everything that is not part of any type of equation environment (like
% |$\ldots$|, 'eqnarray`, etc.), is treated as plain text.
% \emph{Only} the \LaTeX\ commands that are present in the following list, may
% be used here. The ones in the list are either translated into their respective
% HTML entities or simply discarded (i.e. deleted).
% 
% \begin{itemize}
%     \item \emph{Translated} are:
%         \begin{itemize}
%             \item |\\|, |~|
%             \item |\emph{...}|, |\textit{...}|, |\textbf{...}|,
%             |\underline{...}|, |\(...\)|
%             \item |\begin{center}...\end{center}|
%             \item |\{|, |\}|, |\textbackslash|
%             \item umlaute
%         \end{itemize}
%      \item \emph{Discarded} are:
%         \begin{itemize}
%             \item |\vskip|, |\,|
%         \end{itemize}
% \end{itemize}
% 
% These lists may be extended on demand. Just send us an email with your request
% for modification.
% 
% \subsubsection{Images}
% Moodle allows for the types |png|, |jpg|, |gif|, i.e. |eps| and |pdf| may not
% be used.
% 
% \subsubsection{Formulae}
% This refers to symbols that are enclosed by an equation environment  (like
% |$\ldots$|, 'eqnarray`, etc.). The restrictions depend on the plug-in that is
% being used to display formulae. Below, you find the result from our experience
% with mimeTeX.
% 
% \begin{itemize}
%   \item The definition of new macros is only allowed, if they do not take
%   parameters (we do not replace \#1 and the like).  Also, this feature is still
%   in beta stage.
%   \item Don't use references.
%   \item Don't use |\makebox|
% \end{itemize}
% 
% %\subsubsection{Grading}
% Since there is no possibility to specify additional parameters during
% question import into Moodle, we had to define a standard here:\\
% The \verb|\question| macro distributes the full $100\%$ equally among its true
% answers, while false answers count as $-100\%$ in order to punish 
% guessing.\\
% The \verb|\questionSc| macro does not punish guessing; false answers
% count as $0\%$.
% 
% \StopEventually{\PrintIndex}
%
% \section{Implementation}
%
%    \begin{macrocode}
%<*package>
%    \end{macrocode}
%  
%  
%  
%    \begin{macrocode}
\newcommand{\toNemFileNoArg}[1]{}
\newcommand{\toNemFile}[2]{}
\newcommand{\generateNemFile}{%
    \newwrite\nemesisWrite%
    \immediate\openout\nemesisWrite=\jobname.nem%
    \newtoks\nemesisToks%
    \renewcommand{\toNemFileNoArg}[1]{%
        \immediate\write\nemesisWrite{##1}%
        \immediate\write\nemesisWrite{0}%
    }%
    \renewcommand{\toNemFile}[2]{%
        \immediate\write\nemesisWrite{##1}%
        \immediate\write\nemesisWrite{1}%
        \nemesisToks={##2}%
        \immediate\write\nemesisWrite{\the\nemesisToks}%
    }%
}

\newcommand{\keepme}[1]{\toNemFile{keepme}{#1}{#1}}
\newcommand{\intro}[1]{\toNemFile{intro}{#1}{#1}}

\newcounter{questionOrdinal}
\setcounter{questionOrdinal}{0}

\newcounter{answerOrdinal}
\setcounter{answerOrdinal}{0}

\newcommand{\question}[1]{%
    \addtocounter{questionOrdinal}{1}%
    \setcounter{answerOrdinal}{0}%
    \toNemFile{question}{#1}%
    \styleQuestion{#1}%
}

\newcommand{\questionSc}[1]{%
    \addtocounter{questionOrdinal}{1}%
    \setcounter{answerOrdinal}{0}%
    \toNemFile{questionSc}{#1}%
    \styleQuestionSc{#1}%
}

\newcommand{\true}[1]{%
    \addtocounter{answerOrdinal}{1}%
    \toNemFile{true}{#1}%
    \styleTrue{#1}%
}

\newcommand{\false}[1]{%
    \addtocounter{answerOrdinal}{1}%
    \toNemFile{false}{#1}%
    \styleFalse{#1}%
}

\newcommand{\feedback}[1]{%
    \toNemFile{feedback}{#1}%
    \styleFeedback{#1}%
}

\newcommand{\explanation}[1]{%
    \toNemFile{explanation}{#1}%
    \styleExplanation{#1}%
}


\newcommand{\hidesolution}{%
    \renewcommand{\feedback}[1]{%
        \toNemFile{feedback}{##1}%
    }%
    \renewcommand{\explanation}[1]{%
        \toNemFile{explanation}{##1}%
    }%
    \renewcommand{\styleTrue}{%
        \styleTrueHidden%
    }%
    \renewcommand{\styleFalse}{%
        \styleFalseHidden%
    }%
    \renewcommand{\styleDunno}{%
        \styleDunnoHidden%
    }%
}


\newcommand{\styleQuestion}[1]{#1}
\newcommand{\styleQuestionSc}[1]{#1}
\newcommand{\styleTrue}[1]{#1}
\newcommand{\styleFalse}[1]{#1}
\newcommand{\styleFeedback}[1]{#1}
\newcommand{\styleExplanation}[1]{#1}

\newcommand{\styleTrueHidden}[1]{#1}
\newcommand{\styleFalseHidden}[1]{#1}

\newcommand{\styleSolutionQuestion}[1]{#1}
\newcommand{\styleSolutionQuestionSc}[1]{#1}

\newcommand{\styleTrueSol}[1]{#1}
\newcommand{\styleFalseSol}[1]{#1}
\newcommand{\styleSolutionExplanation}[1]{#1}
\newcommand{\styleSolutionFeedback}[1]{#1}


\renewcommand{\styleQuestion}[1]{%
    \bigskip%
    \filbreak%
    \noindent {\bf\arabic{questionOrdinal}.\ }{#1}%
}

\renewcommand{\styleQuestionSc}{\styleQuestion}

\renewcommand{\styleTrue}[1]{%
    \begin{itemize}%
        \item[\begin{tabular}{rr}$\surd$&(\alph{answerOrdinal})\end{tabular}]{#1}
    \end{itemize}%
}

\renewcommand{\styleFalse}[1]{%
    \begin{itemize}%
        \item[\begin{tabular}{rr}&(\alph{answerOrdinal})\end{tabular}]{#1}
    \end{itemize}%
}

\renewcommand{\styleFeedback}[1]{%
    \begin{itemize}%
        \item[]{\par{\footnotesize{#1}}}%
    \end{itemize}%
}

\renewcommand{\styleFalseHidden}[1]{%
    \begin{itemize}%
        \item[\begin{tabular}{rr}&(\alph{answerOrdinal})\end{tabular}]{#1}
    \end{itemize}%
}

\renewcommand{\styleTrueHidden}[1]{%
    \begin{itemize}%
        \item[\begin{tabular}{rr}&(\alph{answerOrdinal})\end{tabular}]{#1}
    \end{itemize}%
}

\renewcommand{\styleSolutionQuestion}[1]{%
    \bigskip%
    \filbreak%
    \noindent {\bf\arabic{questionOrdinal}.\ }%
    {\scriptsize {#1}}%
}

\renewcommand{\styleSolutionQuestionSc}{\styleSolutionQuestion}

\renewcommand{\styleTrueSol}[1]{%
    \begin{itemize}%
        \item[%
            \begin{tabular}{rr}%
                $\surd$&(\alph{answerOrdinal})%
            \end{tabular}%
            ]%
            {\scriptsize {#1}}
    \end{itemize}%
}

\renewcommand{\styleFalseSol}[1]{%
    \begin{itemize}%
        \item[%
            \begin{tabular}{rr}%
                &(\alph{answerOrdinal})%
            \end{tabular}%
            ]%
            {\scriptsize {#1}}
    \end{itemize}%
}

\renewcommand{\styleSolutionExplanation}[1]{%
    \par\noindent {#1}%
}

\renewcommand{\styleSolutionFeedback}[1]{%
    \begin{itemize}%
        \item[]{\par{#1}}%
    \end{itemize}%
}
%    \end{macrocode}
%
%    \begin{macrocode}
%</package>
%    \end{macrocode}
%
%
% \Finale
